\documentclass[10pt]{article} 

\usepackage{hyperref}

% Muliple rows
\usepackage{multirow}

% Margins
\usepackage[margin=2.5cm]{geometry}

% Uncomment for 1 in margins
%\usepackage[in]{fullpage}

\usepackage[usenames,dvipsnames]{color}

% Font stuff
\usepackage{palatino}


\usepackage{listings}
% Code listings
\lstset{
	language=C, 
    basicstyle=\small\ttfamily,
	keywordstyle=\color{black}\bfseries,
	commentstyle=\color{OliveGreen},
}

\begin{document}
\title{ENEL429 Project Coding Standards: DRAFT}
\author{Benjamin Washington-Yule et al.}
\maketitle

\section{Introduction}
This document goes over a few high level coding standards to make sure we can all work with each others code. It is \emph{by no means} declaring that this is the way we have do things, but I hope it will serve as a starting point so we can each have our say on what our opinions are on coding style.

Please tell me what your feelings are on these matters. Especially anything you'd like to change/add that you feel is important.

\section{Revision Control}
An svn repository has been set up already, but I know that the computer engineers of the group would rather use git. I have no experience with git, but I'm not too fussed about which we use, although I do wonder about the benefits we'd gain from using git over svn for such a small project.

It looks to me like we have three options:
\begin{enumerate}
  \item Ditch svn (if we are allowed) and use git. The onus would be on the computer engineers to teach the rest of us how to use the new system.
  \item Stick with svn. I have a feeling Mike has given himself access to our repo for a reason: he wants to keep an eye on what we are doing. Will he be able to do this if we move to git?
  \item Employ something like git-svn so that those who want to use git can do so, while the others can use svn. See \url{http://ecewiki.elec.canterbury.ac.nz/mediawiki/index.php/Git}. Of note is the passage: \emph{``Git-svn allows Git to interface with a subversion repository. This is useful if you already have a project that uses subversion."}
\end{enumerate}

\section{Coding Style}
With six people, it is more than likely that there will be six individual coding styles. Probably the easiest rule is simply to follow the style of whoever created or owns the module. That said, we can lay down some basic style rules from the outset. Personally, I tend to default to whatever Mike Hayes uses\footnote{Except for operating system choice.} as it is clear and makes sense to me. Also, he is the one marking our code,

The norm for (embedded) C seems to be \texttt{underscore\_separators} rather than \texttt{camelCase}. If anyone does want to use camelCase (and they own the module), it is important to call variable/function names the same as they would be had they been written with underscores. E.g. \texttt{pio\_config\_set(...)} becomes \texttt{pioConfigSet(...)} et cetera.

To allow for text editors like vim to easily open multiple file at once, all
code should not exceed the 80th character column. However there are situations
where this rule should not be abided by. If wrapping the code will make it
dramatically harder to read, ignore this rule.

Other things like the position of the opening brace for loops and finicky white-space issues, for example: ``\texttt{my\_way()}" versus ``\texttt{their\_way ()}" are probably best dealt with by following the golden rule of sticking with the existing style, decided on by whoever owns the module.

\subsection{Environments and Layouts}
We will all be using very different development environments and we must be able to easily browse each others code. To make life easier on everyone we can adopt a maximum line length of 80 characters, anything longer should be manually broken. Luckily C is not sensitive to whitespace so this is relatively easy using the backslash character.
\begin{lstlisting}[frame=single,showstringspaces=false]
/* A function broken over two lines. */
static void rblack_rating (int annoyance_level, int talent_level \
                             int attractiveness)
{
    printf("Rebecca Black is the new queen of \\b\\\n");
    ...
}
\end{lstlisting}
Also because of the mix of operating systems (GNU/Linux, BSD, Darwin, Windows) we need to utilise a common line ending.  Because of its long standing value in programming environments this should be Unix (LF or {\texttt\textbackslash n}) endings.

\subsection{Naming Conventions}
It is important that we get these correct so that we know if we are dealing with a structure, function or variable and which module it belongs to.
\subsubsection{Functions}
Mike Hayes uses the following style: \texttt{<module\_name>-<verb>}, or \texttt{<module\_name>-<property>-<verb>}. Examples of these are:
\begin{itemize}
  \item \texttt{pio\_config\_set(...)}
  \item \texttt{spi\_eeprom\_write(...)}
  \item \texttt{spi\_mode\_set(...)}
  \item \texttt{adc\_read(...)}
\end{itemize}
At the very minimum I suggest we prefix functions with the module name.
\subsubsection{Structures}
There's three naming options here: \texttt{NamingExample}, \texttt{Naming\_Example} and \texttt{naming\_example\_t}. Personally I like the latter but I think the last two are clear enough to be recognised immediately (the first may clash with anyone using camelCase). It may also may life easier if the structure is typedef'd to a pointer to that structure.

\begin{lstlisting}[frame=single]
/* Naming method 1: */
struct example_struct {
	uint8_t something;
	bool something_else;
};

typedef example_struct *example_t;

/* Usage: */

example_t instance = example_init(); // Returns a pointer to a struct.
\end{lstlisting}
\begin{lstlisting}[frame=single]
/* Naming method 2: */
struct Example_Struct {
	uint8_t something;
	bool something_else;
};

typedef Example_Struct *Example;

/* Usage: */

Example instance = example_init(); // Returns a pointer to a struct.
\end{lstlisting}

\subsubsection{Enumerations}
To avoid clashing with each others enumerations, it may help to prefix the module name to the enumeration, especially if an element of the enum is a common word. Example:
\begin{lstlisting}[frame=single]
enum motor_speed_enum {MOTOR_SLOW = 0, MOTOR_NORMAL, MOTOR_FAST};

typedef enum motor_speed_enum motor_speed_t;
\end{lstlisting}

\subsection{Documentation}
The use of Doxygen throughout modules is an excellent idea. At the very least, functions should have a one-line comment about what they do. 

\begin{lstlisting}[frame=single]
/* Good: */

/* Set the current motor speed out of 255 */
bool motor_speed_set(motor_t m, uint8_t speed);
\end{lstlisting}

\begin{lstlisting}[frame=single]
/* Better: */

/** Set the current motor speed. 
	@param m The motor object.
	@param speed A number from 0 to 255. 0 = stopped, 255 = full speed.
	@return 1 on success, 0 on failure. */
bool motor_speed_set(motor_t m, uint8_t speed);
\end{lstlisting}

\section{SCR's input}
I'm on team Benjamin for all of the above. I definitely prefer embedded C looking like 'noun\_verb(args)' than any of that java-esque shite, typedefs ending in \_t and keeping the namespace clean from obvious enum names. 
I have a couple of things to add, some of which I actually think are semi-important, and some of which are more software than syntax:
\begin{list}{*}{}
	\item Always brace blocks (ie even if it's just one line).
	\item No nested comments.
	\item Temporarily remove large blocks of code only using preprocessor blocks.
	\item One action per line (more of a suggestion than a guideline however assignments in conditional statements make baby jesus cry: 'if ((date = getDate()) == FRIDAY){...}'
	\item No reliance on operator precedence.
	\item Heavy handed encapsulation, which pretty much just comes down to static-ing everything you don't want exposed outside of you module.
\end{list}

\section{Final Words}
What say you? I'd like to know what you agree with and disagree with. Especially around coding style. It's almost guaranteed I've left something important out too.

\end{document}
